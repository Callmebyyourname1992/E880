
\documentclass[11pt]{article}
\usepackage[margin=1in]{geometry}
\usepackage{caption} %For captioning objects
\usepackage{subcaption} %sub-captioning pictures
\usepackage{graphicx} %to include graphics
\usepackage{hyperref} %for clickable references
\usepackage{listings} %to write code listings
\lstset{language=R, breaklines=true}  
\usepackage[mathcal]{euscript} %for curly S
\usepackage{mathtools}
\usepackage{float} %so figures can be placed "here"

%Defining commands for math symbols
\usepackage{amsmath} %to enable split equations
\usepackage{statmath} %for plim. Be careful, it has already \E and \V.
\usepackage{amssymb} %to enable mathbb
\newcommand{\R}{\mathtt{R}} %Software R
\renewcommand{\E}{\mathbb{E}} %expectation
\usepackage{bbm}
\newcommand{\1}{\mathbbm{1}}
\renewcommand{\V}{\mathbb{V}} %variance
\newcommand{\N}{\mathcal{N}} %normal distribution
\newcommand{\U}{\mathcal{U}} %normal distribution
\renewcommand{\P}{\mathbb{P}} %proba, renewcom since \P already exists
%Regression variables in vector form
\newcommand{\y}{\boldsymbol{y}} 
\newcommand{\x}{\boldsymbol{x}} 
\newcommand{\z}{\boldsymbol{z}} 
\newcommand{\yhat}{\boldsymbol{\hat{y}}} 
\renewcommand{\u}{\boldsymbol{u}} 
\newcommand{\uhat}{\boldsymbol{\hat{u}}}  
\newcommand{\Px}{\boldsymbol{P}}  
\newcommand{\Mx}{\boldsymbol{M}}  
\newcommand{\A}{\boldsymbol{A}}  
\newcommand{\X}{\boldsymbol{X}}  
\newcommand{\Z}{\boldsymbol{Z}}  
\newcommand{\e}{\boldsymbol{e}}  
\renewcommand{\r}{\tilde{r}}  
%\renewcommand{\r}{\boldsymbol{r}} 
\renewcommand{\i}{\boldsymbol{\imath}} 
\newcommand{\alphab}{\boldsymbol{\alpha}}  
\newcommand{\betab}{\boldsymbol{\beta}}  
%opening

\title{Problem Set 1 - ECON 880\\
	\small Spring 2022 - University of Kansas}
\author{Minh Cao, Gunawan}


\begin{document}
<<<<<<< HEAD


\begin{center}
{\Large Econ 880 \hspace{0.5cm} HW 1}\\
\textbf{Minh Cao, Gunawan}\\ %You should put your name here
Due:  %You should write the date here.
\end{center}

\vspace{0.2 cm}


\subsection*{Exercises for chapter 1 and 2}

\begin{enumerate}
\item Question 1 
\item We can evaluate the polynomial as follow:\\

 $$ f(x,y) = 83521y^8+578x^{2}y^{4} -2x^{4}+2x^{6}-x^{8} = y^{4}(83521y^{4}+578x^{2})+x^{4}(-2+2x^{2}-x^{4})$$
 
 The good thing about Honer's method are:\\
 
 \item Consider the original way:
 \item The Honer's way

eneej


\end{enumerate}

\end{document}kdjkdfjfr
=======
\maketitle	
\section*{Problem 1}
\subsection*{Problem 1a}
We can evaluate the polynomial as follow:\\
\begin{eqnarray*}
f(x,y) 	&=& 83521y^8+578x^{2}y^{4} -2x^{4}+2x^{6}-x^{8}\\
 		&=& y^{4}(83521y^{4}+578x^{2})+x^{4}(-2+2x^{2}-x^{4})
\end{eqnarray*}
\subsection*{Problem 1b}
	Test
\subsection*{Problem 1c}
	
\section*{Problem 2}
	
\section*{Problem 3}
	
\section*{Problem 4} Minh update.....
	
\end{document}
>>>>>>> 0c2eaa1d59bb50b0557d17ef593ea9a45318db2b
