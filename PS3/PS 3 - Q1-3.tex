
\documentclass[11pt]{article}
\usepackage[margin=1in]{geometry}
\usepackage{caption} %For captioning objects
\usepackage{subcaption} %sub-captioning pictures
\usepackage{graphicx} %to include graphics
\usepackage{hyperref} %for clickable references
\usepackage{listings} %to write code listings
\lstset{language=R, breaklines=true}  
\usepackage[mathcal]{euscript} %for curly S
\usepackage{mathtools}
\usepackage{float} %so figures can be placed "here"

%Defining commands for math symbols
\usepackage{amsmath} %to enable split equations
\usepackage{statmath} %for plim. Be careful, it has already \E and \V.
\usepackage{amssymb} %to enable mathbb
\newcommand{\R}{\mathtt{R}} %Software R
\renewcommand{\E}{\mathbb{E}} %expectation
\usepackage{bbm}
\newcommand{\1}{\mathbbm{1}}
\renewcommand{\V}{\mathbb{V}} %variance
\newcommand{\N}{\mathcal{N}} %normal distribution
\newcommand{\U}{\mathcal{U}} %normal distribution
\renewcommand{\P}{\mathbb{P}} %proba, renewcom since \P already exists
%Regression variables in vector form
\newcommand{\y}{\boldsymbol{y}} 
\newcommand{\x}{\boldsymbol{x}} 
\newcommand{\z}{\boldsymbol{z}} 
\newcommand{\yhat}{\boldsymbol{\hat{y}}} 
\renewcommand{\u}{\boldsymbol{u}} 
\newcommand{\uhat}{\boldsymbol{\hat{u}}}  
\newcommand{\Px}{\boldsymbol{P}}  
\newcommand{\Mx}{\boldsymbol{M}}  
\newcommand{\A}{\boldsymbol{A}}  
\newcommand{\X}{\boldsymbol{X}}  
\newcommand{\Z}{\boldsymbol{Z}}  
\newcommand{\e}{\boldsymbol{e}}  
\renewcommand{\r}{\tilde{r}}  
%\renewcommand{\r}{\boldsymbol{r}} 
\renewcommand{\i}{\boldsymbol{\imath}} 
\newcommand{\alphab}{\boldsymbol{\alpha}}  
\newcommand{\betab}{\boldsymbol{\beta}}  
%opening

\newcounter{daggerfootnote}
\newcommand*{\daggerfootnote}[1]{%
	\setcounter{daggerfootnote}{\value{footnote}}%
	\renewcommand*{\thefootnote}{\fnsymbol{footnote}}%
	\footnote[2]{#1}%
	\setcounter{footnote}{\value{daggerfootnote}}%
	\renewcommand*{\thefootnote}{\arabic{footnote}}%
}


\title{Problem Set 3 - ECON 880\\
	\small Spring 2022 - University of Kansas}
\author{Gunawan, Minh Cao}


\begin{document}

\maketitle	

\section*{Problem 1}
In this exercise, we are interested in solving $Ax=b$, where
\[A = \begin{pmatrix}
	54 &14& -11& 2 \\ 14 &50& -4& 29 \\ -11 &-4 &55& 22 \\ 2& 29& 22& 95
\end{pmatrix}, \quad b = \begin{pmatrix}
	1\\1\\1\\1
\end{pmatrix} \]
using Gauss-Jacobi and Gauss-Seidel method. Both methods yield the same result
\[x = \begin{pmatrix}
	0.0189\\
	0.0168\\
	0.0234\\
	-0.0004
\end{pmatrix}.\]
Gauss-Jacobi method required 0.0129388 seconds with 45 iterations until convergence. The residual is given by
\[
10^{-11} \times \begin{pmatrix}
	0.036104452760810\\
	-0.121147536447097\\
	-0.092215124425365\\
	0.235056418773638
\end{pmatrix}
\]
Gauss-Seidel method required 0.0176893 seconds with 23 iterations until convergence. The residual is given by
\[10^{-12}\times \begin{pmatrix}
	0.319078097277270\\
	-0.579647441156794\\
	-0.376587649952853\\
	0
\end{pmatrix}\]
   
\section*{Problem 2}
In this exercise, we are interested in solving $Bq=r$ using extrapolation, where

\[B = \begin{pmatrix}
 1&0.5&0.3\\0.6&1&0.1\\0.2&0.4&1
\end{pmatrix}, \quad r = \begin{pmatrix}
	5\\7\\4
\end{pmatrix}. \]
Following Ken Judd's definition\daggerfootnote{Kenneth L. Judd, 1998. "Numerical Methods in Economics," MIT Press Books, The MIT Press, p.78-79}, we first define $G=I-B$, and run the following iteration
\[q^{k+1}=\omega G q^k +\omega r + (1-\omega)q^k,\] 
where we pick $\omega=1.05$, tolerance level $10^{-13}$, and initial value $q_0=(0,0,0)'$. The extrapolation converged after $k=148$ iterations, with the residual 
\[Bq-r =  10^{-12} \times
\begin{pmatrix}
0.319078097277270\\
-0.579647441156794\\
-0.376587649952853
\end{pmatrix}.\] 
The solution to the linear equation system is
\[
q=\begin{pmatrix}
	1.6716\\
	5.8651\\
	1.3196
\end{pmatrix}
\]
\section*{Problem 3}
We want to solve the following functions 
\begin{enumerate}
	\item $\sin(2\pi x)-2x=0$
	\item $\sin(2\pi x)-x=0$
	\item $\sin(2\pi x)-0.5x=0$
\end{enumerate}
using 1) Bisection, 2) Newton method, 3) Secant method, and 4) fixed-point iteration. We want to evaluate for what value of initial guess $x_0\in[-2,2]$ these methods converge. We proceed by first plotting all the three functions on Figure \ref{fig:3}. From these graphs, we see that within the interval $[-2,2]$, function 1 and 2 have both two roots, while function 3 has seven roots.

\begin{figure}[htbp]
	\centering
	\begin{subfigure}[b]{0.48\textwidth}
		\includegraphics[width=\textwidth]{f1.png}
		\caption{$f(x)=\sin(2\pi x)-2x$}
		\label{3a}
	\end{subfigure}
	~
	\begin{subfigure}[b]{0.48\textwidth}
		\includegraphics[width=\textwidth]{f2.png}
		\caption{$f(x)=\sin(2\pi x)-x$}
		\label{3c}
	\end{subfigure}
	\newline
	\begin{subfigure}[b]{0.48\textwidth}
		\includegraphics[width=\textwidth]{f3.png}
		\caption{$f(x)=\sin(2\pi x)-0.5x$}
		\label{3e}
	\end{subfigure}
	\caption{Function plots for Problem 3}
	\label{fig:3}
\end{figure}

\subsection*{3(a) Bisection}
In order for Bisection to work, we need to pick two values $x_{low}$ and $x_{high}$ so that $f(x_{low})\cdot f(x_{high})<0$. The range of admissible values for $x$ for the three functions above is summarized in Table \ref{tab:3:1}
	\begin{table}[h]
	\centering
	\begin{tabular}{l c c c }
		\hline
		\hline
		Function & Root &\multicolumn{2}{ c }{Admissible Range} \\
		&&$x_{low}$& $x_{high}$\\
		\hline
		$\sin(2\pi x)-2x=0$	& $x_1=-0.368$&$[-2,-0.368) $&$(-0.368,0)$\\
							& $x_2=0$&$(-0.368,0) $&$(0,0.368)$\\
							& $x_3=0.368$&$(0,0.368) $&$(0.368,2]$\\
		\hline
		$\sin(2\pi x)-x=0$	& $x_1=-0.429$&$[-2,-0.429) $&$(-0.429,0)$\\
							& $x_2=0$&$(-0.429,0) $&$(0,0.429)$\\
							& $x_3=0.429$&$(0,0.429) $&$(0.429,2]$\\
		\hline
		$\sin(2\pi x)-0.5x=0$	& $x_1=-1.379$&$[-2,-1.379) $&$(-1.379,-1.092)$\\
								& $x_2=-1.092$&$(-1.379,-1.092)$&$(-1.092,-0.463)$\\
								& $x_3=-0.463$&$(-1.092,-0.463)$&$(-0.463,0)$\\	
								& $x_4=0$&$(-0.463,0)$&$(0,0.463)$\\	
								& $x_5=0.463$&$(0,0.463) $&$(0.463,1.092)$\\	
								& $x_6=1.092$&$(0.463,1.092) $&$(1.092,1.379)$\\	
								& $x_7=1.379$&$(1.092,1.379)$&$(1.379,2]$\\		
		\hline
		\hline
	\end{tabular} 
	\caption{Polynomial evaluation costs}
	\label{tab:3:1}
\end{table}

\subsection*{3(b) Newton Method}

\end{document}

